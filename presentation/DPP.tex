\documentclass{beamer}
\setbeamertemplate{navigation symbols}{}
\usepackage{beamerthemeshadow}

\beamersetuncovermixins{\opaqueness<1>{25}}{\opaqueness<2->{15}}
\begin{document}
\title{Entwurfsmuster in dynamischen Sprachen}
\author{Nick Zbinden und Michael Sprecher}
\date{\today}

\begin{frame}
\titlepage
\end{frame}

\begin{frame}\frametitle{Wie und Warum}
\begin{center}
  \begin{tabular}{l  l}
    Wer? & Wie?\\
    \hline
    Sprecher, Michael & Python\\
    Zbinden, Nick & Clojure\\
  \end{tabular}

  \paragraph{warum}

    Um zu zeigen wie Pattern in verschiedenen Sprachen
    unterschiedlich implementiert und genützt werden finden wir es
    wichtig, dass nicht nur statisch typisiert Objektorientierte
    Sprachen angeschaut werden sondern auch dynamische oder funktionale.
\end{center}
\end{frame}

%Strategie Pattern


  %Clojure
\begin{frame}
  
\end{frame}
  %End Clojure

  %Python 


  %End Python

  %Zusammenfassungsslide
%End

%Strategie Observer


  %Clojure



  %End Clojure

  %Python 


  %End Python

  %Zusammenfassungsslide

%End

%Strategie Decorater
  %Clojure

  %End Clojure

  %Python 

  %End Python

  %Zusammenfassungsslide
%End


%Strategie MVC
  %Clojure

  %End Clojure

  %Python 

  %End Python

  %Zusammenfassungsslide
%End
\end{document}
