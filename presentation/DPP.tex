\documentclass{beamer}
\setbeamertemplate{navigation symbols}{}
\usepackage{beamerthemeshadow}
\usepackage{listings}

\beamersetuncovermixins{\opaqueness<1>{25}}{\opaqueness<2->{15}}

\lstset{ 
language=Lisp,                  % choose the language of the code
basicstyle=\footnotesize,       % the size of the fonts that are used for the code
numbers=left,                   % where to put the line-numbers
numberstyle=\footnotesize,      % the size of the fonts that are used for the line-numbers
stepnumber=1,                   % the step between two line-numbers. If it's 1 each line 
                                % will be numbered
numbersep=5pt,                  % how far the line-numbers are from the code
backgroundcolor=\color{white},  % choose the background color. You must add \usepackage{color}
showspaces=false,               % show spaces adding particular underscores
showstringspaces=false,         % underline spaces within strings
showtabs=false,                 % show tabs within strings adding particular underscores
frame=single,	                % adds a frame around the code
tabsize=2,	                % sets default tabsize to 2 spaces
captionpos=b,                   % sets the caption-position to bottom
breaklines=true,                % sets automatic line breaking
breakatwhitespace=false,        % sets if automatic breaks should only happen at whitespace
title=\lstname,                 % show the filename of files included with \lstinputlisting;
                                % also try caption instead of title
escapeinside={\%*}{*)},         % if you want to add a comment within your code
morekeywords={defn, def}            % if you want to add more keywords to the set
}

\begin{document}
\title{Entwurfsmuster in dynamischen Sprachen}
\author{Nick Zbinden und Michael Sprecher}
\date{\today}

\begin{frame}
  \titlepage
\end{frame}

\begin{frame}\frametitle{Wie und Warum}
\begin{center}
  \begin{tabular}{l  l}
    Wer? & Wie?\\
    \hline
    Sprecher, Michael & Python\\
    Zbinden, Nick & Clojure\\
  \end{tabular}

    Um zu zeigen wie Pattern in verschiedenen Sprachen
    unterschiedlich implementiert und gen�tzt werden finden wir es
    wichtig, dass nicht nur statisch typisiert Objektorientierte
    Sprachen angeschaut werden sondern auch dynamische oder funktionale.
\end{center}
\end{frame}

%Strategie Pattern

  %Clojure
\begin{frame}
  \begin{enumerate}
  \item hash-maps anstelle von klassen \pause
  \item funktionen als ``b�rger erster klasse'' \pause
  \end{enumerate}
\end{frame}

\begin{frame}
  \begin{enumerate}
  \item Strategien: anstelle von Objekte ben�tzen wir Funktionen
    \pause

  \item Enten: anstelle von ben�tzen wir hash-maps \pause
  \item Keine Basisklasse nur Funktionen die Strategien ausf�hren \pause
  \end{enumerate}
\end{frame}


  %End Clojure

  %Python 


  %End Python

  %Zusammenfassungsslide
%End

%Strategie Observer


  %Clojure



  %End Clojure

  %Python 


  %End Python

  %Zusammenfassungsslide

%End

%Strategie Decorater
  %Clojure

  %End Clojure

  %Python 

  %End Python

  %Zusammenfassungsslide
%End


%Strategie MVC
  %Clojure

  %End Clojure

  %Python 

  %End Python

  %Zusammenfassungsslide
%End
\end{document}
