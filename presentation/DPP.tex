\documentclass[compress, red]{beamer}
\mode<presentation>
\usetheme{Warsaw} % Beamer Theme
% other themes: Warsaw, AnnArbor, Antibes, Bergen, Berkeley, Berlin, Boadilla, boxes, CambridgeUS, Copenhagen, Darmstadt, default, Dresden, Frankfurt, Goettingen,
% Hannover, Ilmenau, JuanLesPins, Luebeck, Madrid, Maloe, Marburg, Montpellier, PaloAlto, Pittsburg, Rochester, Singapore, Szeged, classic

\usecolortheme{orchid} % Beamer Color Theme
% color themes: albatross, beaver, beetle, crane, default, dolphin, dov, fly, lily, orchid, rose, seagull, seahorse, sidebartab, structure, whale, wolverine

%\hypersetup{pdfpagemode=FullScreen} % makes your presentation go automatically to full screen

%\useoutertheme[subsection=false]{smoothbars} % Beamer Outer Theme
%\useinnertheme{rectangles} % Beamer Inner Theme

\usepackage{listings}
\beamersetuncovermixins{\opaqueness<1>{25}}{\opaqueness<2->{15}}

%\lstset{ 
%language=Octave,                  % choose the language of the code
%basicstyle=\footnotesize,       % the size of the fonts that are used for the code
%numbers=left,                   % where to put the line-numbers
%numberstyle=\footnotesize,      % the size of the fonts that are used for the line-numbers
%stepnumber=1,                   % the step between two line-numbers. If it's 1 each line 
                               % will be numbered
%numbersep=5pt,                  % how far the line-numbers are from the code
%backgroundcolor=\color{white},  % choose the background color. You must add \usepackage{color}
%showspaces=false,               % show spaces adding particular underscores
%showstringspaces=false,         % underline spaces within strings
%showtabs=false,                 % show tabs within strings adding particular underscores
%frame=single,	                % adds a frame around the code
%tabsize=2,	                % sets default tabsize to 2 spaces
%captionpos=b,                   % sets the caption-position to bottom
%breaklines=true,                % sets automatic line breaking
%breakatwhitespace=false,        % sets if automatic breaks should only happen at whitespace
%title=\lstname,                 % show the filename of files included with \lstinputlisting;
                                % also try caption instead of title
%escapeinside={\%*}{*)},         % if you want to add a comment within your code
%}

%\setbeamercovered{dynamic}

\begin{document}
\title{Entwurfsmuster in dynamischen Sprachen}
\subtitle{Ein vergleich von Java, Python und Clojure}
\author{Nick Zbinden und Michael Sprecher}
\date{\today}

\begin{frame}
  \titlepage
\end{frame}

\begin{frame}\frametitle{Wie und Warum}
\begin{center}
  %\begin{tabular}{l  l}
  %  Wer? & Wie?\\
  %  \hline
  %  Sprecher, Michael & Python\\
  %  Zbinden, Nick & Clojure\\
  %\end{tabular}
  %\vspace{1cm}

\begin{block}{Warum:}
  Um zu zeigen, dass es noch etwas anderes gibt als statisch
  typisierte objektorientiert Sprachen.
\end{block}

%\begin{block}{Vorlage: Java}
%  \begin{itemize}
%    \item limitier statisch typisiert
%    \item classenbasiertes OO
%    \item kaum support für FP
%  \end{itemize}
%\end{block}

\begin{block}{Michael: Python}
  \begin{itemize}
    \item dynamisch typisiert
    \item classenbasiertes OO
    \item limiter support für FP
  \end{itemize}
\end{block} 

\begin{block}{Nick: Clojure}
  \begin{itemize}
    \item dynamisch
    \item OO durch multimethods und records
    \item starker support für FP
  \end{itemize}
\end{block}

\end{center}
\end{frame}

%Strategie Pattern

  %Clojure

\begin{frame}\frametitle{Strategie Pattern in Clojure}
  \begin{enumerate}
  \item Strategien: anstelle von Objekte benützen wir Funktionen
    \pause
  \item Enten: anstelle von benützen wir hash-maps 
    \pause
  \item Keine Basisklasse nur Funktionen die Strategien ausführen 
    \pause
  \end{enumerate}
\end{frame}

\begin{frame}\frametitle{Strategien = Funktionen}
  
\end{frame}

  %End Clojure

  %Python 

%\begin{frame}\frametitle{Strategie Pattern in Python}
%}


%\end{frame}

  %End Python

  %Zusammenfassungsslide
%End

%Strategie Observer


  %Clojure



  %End Clojure

  %Python 


  %End Python

  %Zusammenfassungsslide

%End

%Strategie Decorater
  %Clojure

  %End Clojure

  %Python 

  %End Python

  %Zusammenfassungsslide
%End

\end{document}
